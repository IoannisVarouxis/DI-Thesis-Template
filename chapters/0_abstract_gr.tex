\abstractGr{

\selectlanguage{greek}

    Τον άντρα, Μούσα, τον πολύτροπο να μου ανιστορήσεις, που βρέθηκε ως τα πέρατα του κόσμου να γυρνά, αφού της Τροίας πάτησε το κάστρο το ιερό. Γνώρισε πολιτείες πολλές, έμαθε πολλών ανθρώπων τις βουλές, κι έζησε, καταμεσής στο πέλαγος, πάθη πολλά που τον σημάδεψαν, σηκώνοντας το βάρος για τη δική του τη ζωή και των συντρόφων του τον γυρισμό. Κι όμως δεν μπόρεσε, που τόσο επιθυμούσε, να σώσει τους συντρόφους. Γιατί εκείνοι χάθηκαν απ᾽ τα δικά τους τα μεγάλα σφάλματα, νήπιοι και μωροί, που πήγαν κι έφαγαν τα βόδια του υπέρλαμπρου Ήλιου· κι αυτός τους άρπαξε του γυρισμού τη μέρα. Από όπου θες, θεά, ξεκίνα την αυτή την ιστορία, κόρη του Δία, και πες την και σ᾽ εμάς. 

    Τότε λοιπόν οι άλλοι, όσοι ξέφυγαν τον άθλιον όλεθρο, όλοι τους ήσαν σπίτι τους, γλιτώνοντας κι απ᾽ του πολέμου κι απ᾽ της θάλασσας τη μάχη. Μόνο εκείνον, που τον παίδευε πόθος διπλός, του γυρισμού και της γυναίκας του, τον έκρυβε κοντά της μια νεράιδα, η Καλυψώ, θεά σεμνή κι αρχοντική, στις θολωτές σπηλιές της, γιατί τον ήθελε δικό της. Κι όταν, με του καιρού τ᾽ αλλάγματα, ο χρόνος ήλθε που του ορίσαν οι θεοί να δει κι αυτός το σπίτι του, να φτάσει στην Ιθάκη, ούτε κι εκεί δεν έλειψαν οι αγώνες, κι ας ήταν πια με τους δικούς του. Ωστόσο οι θεοί τώρα τον συμπαθούσαν, όλοι εκτός του Ποσειδώνα· αυτός σφοδρό κρεμούσε τον θυμό του πάνω στον θεϊκό Οδυσσέα, προτού πατήσει της πατρίδας του το χώμα.

    Εκείνον όμως τον καιρό ο Ποσειδώνας είχε ταξιδέψει στους μακρινούς Αιθίοπες — οι Αιθίοπες στις δύο άκρες του κόσμου μοιρασμένοι· μισοί όπου ο ήλιος βασιλεύει, μισοί απ᾽ όπου ο ήλιος ανατέλλει. Πήγε να πάρει μέρος στη θυσία, μιαν εκατόμβη με ταύρους και κριάρια, και τώρα ευφραίνονταν στις τάβλες καθισμένος. Τότε συνάχτηκαν οι υπόλοιποι θεοί στου ολύμπιου Δία το παλάτι, όπου εκείνος πρώτος πήρε τον λόγο, πατέρας ανθρώπων και θεών. Στον νου του φέρνοντας, θυμήθηκε τον φημισμένο Αίγισθο, που τον θανάτωσε ο ξακουστός Ορέστης, γιος του Αγαμέμνονα· αυτόν θυμήθηκε μιλώντας ο θεός στους αθανάτους: 

    Αλίμονο, είναι αλήθεια ν᾽ απορείς που θέλουν οι θνητοί να ρίχνουν στους θεούς τα βάρη τους· έρχεται λένε το κακό από μας — κι όμως οι ίδιοι, κι από φταίξιμο δικό τους, πάσχουν και βασανίζονται, και πάνω απ᾽ το γραφτό τους. Έτσι και τώρα ο Αίγισθος, την ορισμένη μοίρα παραβαίνοντας, πήγε να σμίξει με τη νόμιμη γυναίκα ενός Ατρείδη, κι αυτόν τον σκότωσε στου γυρισμού την ώρα, γνωρίζοντας τι τιμωρία σκληρή τον περιμένει· αφού εμείς του στείλαμε τον άγρυπνον αργοφονιά Ερμή με μήνυμα, μήτε εκείνον να σκοτώσει μήτε και τη γυναίκα του να μπλέξει σε παράνομο κρεβάτι· αλλιώς θα πέσει στο κεφάλι του η εκδίκηση του γιου για τον πατέρα, όταν ο Ορέστης, παλληκάρι πια, θελήσει να γυρίσει στην πατρίδα. Αυτά, με τόση φρόνηση ο Ερμής μιλώντας, του μηνούσε, κι όμως τον νου του Αιγίσθου δεν κατόρθωσε ν᾽ αλλάξει. Τώρα, ακέριο και μεμιάς, το άνομο κρίμα του ξεπλήρωσε. 

    Αμέσως ανταπάντησε, τα μάτια λάμποντας, η Αθηνά: Πατέρα μας των αθανάτων, Κρονίδη, των δυνατών ο παντοδύναμος, καλά κι όπως του ταίριαζε, εκείνος αφανίστηκε και πάει — την ίδια μοίρα να ᾽χει κι όποιος ανάλογα κριματιστεί. Εμένα όμως για τον Οδυσσέα φλέγεται η καρδιά μου· γενναίος αλλά δύσμοιρος, να βασανίζεται με τόσα πάθη, απ᾽ τους δικούς του χωρισμένος, σ᾽ ένα περίβρεχτο νησί, στον ομφαλό, όπως λένε, της θαλάσσης. Νησί κατάφυτο με δέντρα, και μια θεά το κατοικεί στα δώματά της· η θυγατέρα του Άτλαντα, που η γνώμη του γυρίζει μόνο στο κακό — ξέρει καλά αυτός των θαλασσών τα βάθη, και πάνω του σηκώνει ψηλές κολόνες, να κρατούν τον ουρανό χώρια απ᾽ τη γη. Η θυγατέρα του λοιπόν τον Οδυσσέα κατακρατεί, δύστυχο κι οδυρόμενο· λόγια γλυκά προφέροντας και μαλακά σαν χάδια, τον θέλγει ακατάπαυστα, για να ξεχάσει την Ιθάκη. Εκείνος όμως, βυθισμένος στον καημό του, να δει καπνό της πατρικής του γης ψηλά να ανηφορίζει, απελπισμένος εύχεται τον θάνατο. Εσένα ωστόσο, Δία ολύμπιε, ως πότε αλύγιστη θα μείνει η βουλή σου; Ο Οδυσσέας δεν ήταν που θυσίες σού πρόσφερε στην ευρύχωρη Τροία, πλάι στ᾽ αργίτικα καράβια; Πώς και γιατί τόσος θυμός γι᾽ αυτόν, ω Δία;

%    [από \url{https://www.greek-language.gr/digitalResources/ancient_greek/library/ browse.html?text_id=133} ]


\selectlanguage{english}
}